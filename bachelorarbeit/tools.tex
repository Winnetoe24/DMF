\documentclass[./einleitung.tex]{subfiles}
\begin{document}
\section{Auswahl der verwendeten Technologien}
Ein zentraler Teil einer Architektur ist die Auswahl der verwendeten Technologien.
Diese Technologien sollen die Lösung der Aufgaben einer Software vereinfachen.
\newline
Im DMF müssen folgende Aufgaben gelöst werden:
\begin{enumerate}
\item Modelldatei Parsen und \acrshort{ast} generieren
\item \acrshort{ast} auslesen und verarbeiten
\item Kommunikation mit verschiedenen \acrshort{ide}'s
\item Generieren von Codedateien in verschieden Sprachen
\item Integration mit verschiedenen Build Tools
\end{enumerate}
\subsection{Parser}
Der Parser für das \acrshort{dmf} muss große Dateien wiederholt mit kleinen Änderungen parsen.
Diese Anforderung stammt aus der Notwendigkeit des \acrshort{ast}'s um Syntaktische und Semantische Fehler, sowie die verschiedenen Tokens(siehe Abschnitt \acrshort{lsp}) nach jeder Eingabe an die \acrshort{ide} zu übermitteln.
Hierbei ist Latenz die höchste Priorität, denn die Reaktionsfähigkeit der \acrshort{ide} beeinflusst die Geschwindigkeit mit der Entwickelt werden kann. \\
Zusätzlich muss der Parser auch von jeder anderen Komponente des \acrshort{dmf}'s verwendet werden.
Deshalb ist hier die Einschränkung der anderen Technologien unerwünscht.
\subsubsection{XText}
XText ist ein Framework der Eclipse Foundation.\\
Es bietet die Möglichkeit eine \acrfull{dsl} mit verschiedenen Modellen zu modellieren und Regeln automatisch zu überprüfen.
XText setzt auf Modellierung vieler Bestandteile und generiert andere Komponenten komplett.
Dies ermöglicht eine schnelle Entwicklung wenn die Anforderungen perfekt zu XText passen.
XText schränkt stark ein, wo Anpassungen vorgenommen werden können.
So ist es nicht vorgesehen die \acrshort{lsp}-Server Implementierung anzupassen, obwohl XText nicht alle Features des \acrlong{lsp}-Protokolls unterstützt.
Dateigeneration und die Verarbeitung des AST's müssen auch mit den Java-Interfaces von XText vorgenommen werden.
Dies setzt immer die Verwendung von JVM basierten Sprachen voraus.
Jede JVM-Implementierung benötigt beachtliche Zeit zum Starten weshalb Code Generation immer auf den Start Warten muss. \\
Abschließend waren an XText die nicht funktionierenden Beispiel Projekte, sowie zwingende Entwicklung in Eclipse, sehr abweisend.
Ein Framework welche eine einfache und flexible Entwicklung ermöglichen soll, sollte nicht schwer und nur in einer \acrshort{ide} zu entwickeln sein.
\subsubsection{Treesitter}
Treesitter ist ein Open Source Framework zur Generierung von Parsern. \\
Dabei wird die Grammatik mithilfe einer Javascript \acrlong{api} definiert.
Mithilfe der Treesitter \acrlong{cli} wird aus der Javascript Datei der Parser generiert.
Der generierte Parser nutzt C.
C eignet sich hier sehr gut, da es die höchste Performance und die Möglichkeit es in jeder anderen Sprache zu nutzen bietet.
Das Nutzen von C ist für jede Sprache eine Voraussetzung, um mit dem Betriebssystem zu kommunizieren.
C's größter Nachteil, die manuelle Speicher Verwaltung, wird durch die Generation des Parsers gelöst.
Die bereitgestellten Schnittstellen übergeben Strukturen welche vom Aufrufer verwaltet werden.
\paragraph{Iteratives Parsen}
Ein großes Unterscheidungsmerkmal von Treesitter ist die Möglichkeit iterativ zu parsen.
\newline
 \begin{center}
 \textit{With intelligent [node] reuse, changes match the user’s in
tuition; the size of the development record is decreased; and the performance
 of further analyses (such as semantics) improves.\cite{twagner}}
 \end{center}
Beim iterativen Parsen ist das Ziel den \acrshort{ast} nicht bei jedem Parse-Durchlauf neu zu erstellen, sondern möglichst viel des \acrshort{ast}'s wiederzuverwenden.
Für das Iterative Parsen muss der \acrshort{ast} sowie die bearbeiteten Textstellen an Treesitter übergeben werden.
Die Durchlaufzeit des iterativen Parsedurchlaufs hängt nicht mehr der Länge der kompletten Modelldatei ab, sondern nur von den neuen Terminalsymbolen und Modifikationen im \acrshort{ast}:
 \newline
 \begin{center}
 \textit{Our incremental parsing algorithm runs in O(t + slgN) time for t new terminal symbols and s modification sites in a tree containing N nodes \cite{twagner}}
 \end{center}
\subsubsection{\acrfull{ant}}
\acrshort{ant} ist sehr ähnlich zu Treesitter.
Die größten Unterschiede sind die \acrlong{api}'s zum Schreiben der Grammatiken und die Möglichkeit iterativ zu Parsen.
Zusätzlich unterstützt \acrshort{ant} nur Java, C\# und C++.
Dies zwingt einen in der Wahl der Implementierungssprache ein.

\subsubsection{Auswahl Parser}
Für das \acrshort{dmf} Framework wurde Treesitter verwendet.
Die Exellente Performance sowie die Flexibilität bei der Implementierung der restlichen Komponenten hoben Treesitter von den restlichen Technologien ab.


\subsection{\acrshort{ast} Verarbeiten}
Bei der Verarbeitung des \acrshort{ast}'s müssen verschiedene Regeln abgearbeitet werden und der Inhalt des \acrshort{ast}'s in einem Modell vorbereitet werden.
Essenziell für die Verarbeitung ist die Zusammenarbeit mit den folgenden Komponenten.

Die Auswahl der Technologie für diesen Schritt basiert auf der Auswahl für die folgenden Schritte.

\subsection{Kommunikation mit verschiedenen \acrshort{ide}'s}
Damit ein Framework die Entwicklung nicht einschränkt muss es in verschiedenen \acrfull{ide} genutzt werden können.
Viele IDE's stellen Schnittstellen für Plugins bereit. Dazu zählen Intellij, Eclipse, NeoVim und VSCode.
Jede Schnittstellen ist jedoch unterschiedlich, wodurch die Entwicklung von vielen Verschiedenen Plugins nötig wäre.
\paragraph{\acrfull{lsp}}
Eine einfachere Möglichkeit bietet das \acrfull{lsp}.
Dieses Protokoll bietet die Möglichkeit, dass viele verschiedene \acrshort{ide}'s eine Serverimplementierung nutzen.
Im Fall von Zed und Eclipse lassen sich \acrlong{lsp}-Server sogar ohne jegliche Plugins einbinden.
Wobei hier auf die schlechte Unterstützung des \acrlong{lsp}-Protokolls in Eclipse hingewiesen werden muss.
Intellij und NeoVim nutzen Plugins, um \acrlong{lsp}-Server anzubinden.
VSCode bietet eine \acrlong{api} und einen einfachen \acrlong{lsp}-Client in ein kleines Plugin zu implementieren.
Im \acrlong{lsp}-Server können gebündelt Logik und Protokoll implementiert werden.
\newline\newline
\acrlong{lsp} wird hauptsächlich über die Standard-Eingabe und -Ausgabe oder über einen Server Socket transportiert.
Es wird ein JSON-RPC Format genutzt. % TODO Maybe JSOn als Abkürzung
Der \acrlong{lsp}-Server muss somit JSON, Std-In und Std-Out, sowie Server Sockets unterstützen.

\subsubsection{Typescript}
Von der VSCode Dokumentation wird die Implementierung eines \acrlong{lsp}-Servers in Typescript empfohlen.
Dafür werden Bibliotheken bereitgestellt.
Typescript eignet sich gut, für die JSON Parsing und für die Verwendung von Server Sockets.
Probleme entstehen bei Typescript bei den Themen Performance, Anbindung an den Parser und bei der Fehlerbehandlung.

\subsubsection{Golang}
Golang ist eine Sprache, welche für die Entwicklung von Backends ausgelegt wurde. \\
Es werden die Anforderungen für JSON-Parsing, Std-IO und Server Sockets erfüllt, durch die große Standard Bibliothek erfüllt.
Es gibt keine Bibliothek welche das komplette Protokoll beinhaltet.
Dieses kann jedoch durch die Unterstützung von \acrshort{llm}'s schnelle generiert werden.\\
Golang bietet zusätzlich eine simple Anbindung an den Parser und die Möglichkeit sehr einfach Parallelität einzubauen.
Besonders erwähnenswert ist die Geschwindigkeit eines Golang Programmes und die Startgeschwindigkeit ohne auf Speichersicherheit zu verzichten.

\subsubsection{Java}
Java bietet eine bietet mit "lsp4j" eine Bibliothek zur einfachen Entwicklung.
Bei der Einbindung des Parsers gestalten sich jedoch zusätzliche Herausforderungen da, der Java Code Plattform unabhängig kompiliert wird und Plattform abhängigen Code aufrufen muss.
Java benötigt für die Ausführung eine installierte Instanz der JRE. Die JRE muss nicht nur zusätzlich zum \acrlong{lsp}-Server verwaltet werden, sondern benötigt zusätzlich Zeit zum Starten.
So muss der Entwickler länger warten bis seine Entwicklungsumgebung bereitsteht.


\subsection{Generation von Codedateien in verschiedenen Sprachen}
Ziel des DMFs ist es große Mengen an Sourcecode zu generieren.
Dabei soll den Entwicklern die Wahl zwischen mehreren Zielsprachen gegeben werden.
Diese Generation wird beim Build und damit sehr häufig ausgeführt.
Eine langsame Generation wird jeder Organisation viel Geld kosten.

Die Generation muss somit schnell und Zielsprachen unabhängig sein.
Sie muss auch aus von den Build Tools gestartet werden.

\subsubsection{Golang Templates}
Golang Standardbibliothek bietet die Möglichkeit Templates zu definieren.
Diese Templates werden hauptsächlich für die Generierung von HTML genutzt.
Da sie Golang die Templates nicht nur für HTML, sondern auch für generelle Texte anbietet, können diese auch für jede Zielsprache genutzt werden. \\
Die Anforderungen an einen Webserver (Geschwindigkeit, Resourcen schonend, Simpel) komplementieren die Anforderungen an einen Codegenerator sehr gut. \\
Golang Templates stechen besonders für ihre Integration in \acrshort{ide}'s wie z.B.\ in Intellij heraus.

\subsubsection{Java}
Es gibt mehrere Template Engines für Java.
Einige Beispiele wäre FreeMaker oder Apache Velocity.
Beide sind gut unterstützt und bieten alle nötigen Features für die Generierung von Code.

\subsubsection{Typescript}
Für Typescript gibt es viele Template Engines.
Zu den bekannten gehören Eta, liquidjs und squirrelly.
Sie bieten alle die Möglichkeit verschiedene Zielsprachen zu generieren und können mit nodejs ausgeführt werden.

\subsubsection{Auswahl}
Da Golang eine exzellente Unterstützung in Intellij hat und keine Zusätzliche Installation wie NodeJs oder JRE verwalten muss, fiel meine Wahl auf Golang.
\newline
% Warum sind Installation verwalten nicht so gut
Mit der Wahl für Golang für den Generator, ist auch die Wahl für die Verarbeitung des \acrshort{ast}'s und für den \acrlong{lsp}-Server gefallen.

\subsection{Integration mit verschiedenen Build Tools}
Damit eine Generation während des Buildvorgangs ist essenziell, um sicherzustellen, dass der generierte Code aktuell ist.
Damit der Neugeneration werden auch alle eventuelle Anpassungen in den Dateien überschrieben, wodurch Fehler vermieden werden.

\subsubsection{Maven}
Maven ist ein sehr verbreitetes Build Tool für Java.
Maven unterstützt Plugins, welche während des Builds ausgeführt werden und in der Maven Konfiguration konfiguriert werden können.
Die \acrlong{api}s für Maven Plugins ist in Java geschrieben.
Dieses Plugin muss den Generator aufrufen.
Dies ist möglich, indem die Datei des Generators ausgeführt wird.

\subsubsection{NPM}
NPM ist das führende Build Tool für Typescript Projekte.
NPM unterstützt die Ausführung von Terminal Befehlen.
Der Generator kann somit über das Terminal ausgeführt werden.

% Einfachere Aufruf für Golang Programme 
\end{document}

