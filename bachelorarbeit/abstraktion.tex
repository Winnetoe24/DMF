\documentclass[./einleitung.tex]{subfiles}

\begin{document}
\section{Abstraktion des DMF}
Das DMF basiert auf einer Abstraktion der Datenstrukturen aus mehreren Sprachen.
Dafür wurden die Sprachen Java, Typescript,Python, Golang, Rust und C analysiert.
//TODO Auswahl der Sprachen
\subsection{Analyse}
\subsubsection{Analyse der Typen}
Es wurde analysiert, welche Typen als Referenz oder als Wert als Variablentyp genutzt werden können.
\begin{center}
\begin{tabular}{| c || m{4em} | m{5em} | m{5em} | m{4.5em} | m{4.5em} | m{4.5em} |}
\hline
Typen & Java & Typescript & Python & Golang & Rust & C \\
\hline
Wert & Primitive Typen & Primitive Typen & Primitive Typen & Alle Typen & Alle Typen & Alle Typen \\
\hline
Referenz & Objekte & Objekte, Arrays, Funktionen, Klassen & Alles außer primitive Typen & Explizit & Explizit & Explizit \\
\hline
\end{tabular}
\end{center}
TODO:
Welche Primitive Typen 
Warum Typen die Keine Pritimive Typen sind als Argumente. (Datetime)
\subsection{Analyse von Nullwerten}
Nullwerte sind besonders aus Java bekannt und stellen das Fehlen eines Wertes dar. Es zählt zu der Definition eines Types dazu, zu definieren, ob der Typ Nullwerte erlaubt.
Dies muss auch für Werte und Referenzen evaluiert werden.
\begin{center}
\begin{tabular}{| c || m{4em} | m{5em} | m{5em} | m{4.5em} | m{4.5em} | m{4.5em} |}
\hline
Nullwerte & Java & Typescript & Python & Golang & Rust & C \\
\hline
Wert & nein & nein & ja & nein & Explizit & nein \\
\hline
Referenz & ja & Explizit & ja & ja & Explizit & ja \\
\hline
\end{tabular}
\end{center}

TODO:
Referenz nullable und Argumente nicht nullable

\subsection{Collectiontypen}
Um 1:n- oder n:m-Beziehungen im Datenmodell modellieren zu können wurden drei Collection-Typen aus Java ausgewählt und passende Äquivalente zu finden.
\begin{center}
\begin{tabular}{| c || m{3em} | m{4.5em} | m{5.5em} | m{4.5em} | m{3em} | m{4.5em} |}
\hline
Collectiontypes & Java & Typescript & Python & Golang & Rust & C \\
\hline
List & ja & ja (Array) & ja & ja (slice) & ja & ja (Array) \\
\hline
Set & ja & ja & ja & nein & ja & nein \\
\hline
Map & ja & ja & ja (dictionary)& ja & ja & nein \\
\hline

\end{tabular}
\end{center}

TODO: Erklärung der Typen
Besonderheiten z.B. slice

\subsection{Elemente eines Modells}
Um mit dem DMF Daten in Strukturen verschiedener Programmiersprachen darstellen zu können, müssen auch diese Abstrahiert werden.

Grundvoraussetzung sind die Pr

Es gibt in jeder Programmiersprache ein Konstrukt mit dem mehrere Variablen zusammengefasst werden können. C nannte diese Konstrukte Structs. Diese Structs besitzen keine Identität.

\subsection{Identität einer Instanz in der Datenbank}
Ein Modell im DMF Framework soll in einer Datenbank gespeichert werden können.
Dafür müssen Datenbankschlüssel definiert werden. Ein Schlüssel definiert die Identität eines Zeile in einer Tabelle.
Diese Identität muss auch im Modell abgebildet werden.
Das DMF fügt deshalb den Typen "Entity" hinzu, welcher eine Identität besitzt. Er basiert auf dem Struct und kann somit Argumente, Referenzen und Funktionen beinhalten.




\end{document}