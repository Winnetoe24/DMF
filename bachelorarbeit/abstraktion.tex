\documentclass[./einleitung.tex]{subfiles}

\begin{document}
    \section{Abstraktion des \acrshort{dmf}}
    Das \acrshort{dmf} basiert auf einer Abstraktion der Datenstrukturen aus mehreren Sprachen.
    Diese Abstraktion wurde nach folgender Analyse entwickelt.

    \subsection{Analyse}
    Für die Analyse wurden die Sprachen Java, Typescript, Python, Golang, Rust und C analysiert.
    Die Sprachen wurden spezifisch ausgewählt.
    Java ist weitverbreitet in Enterprise Software.
    Typescript ist die Standardsprache für jegliche Websites und viele Backends.
    Python ist in der Datenanalyse weit verbreitet.
    Durch die Popularität in Umfragen wurde Python miteinbezogen.
    Golang ist eine moderne Alternative für Backends und die Implementierungssprache des \acrshort{dmf}.
    Rust ist die moderne Wahl für `low level' Programmierung.
    C ist die Standardsprache für jede `Foreign-Function-Interfaces' und ist weit verbreitet für ältere `low level' Software.

    \subsubsection{Analyse der Typen}
    Es wurde analysiert, welche Typen als Referenz oder als Wert als Variablentyp genutzt werden können.
    \begin{center}
        \begin{tabular}{| c || m{4em} | m{5em} | m{5em} | m{4.5em} | m{4.5em} | m{4.5em} |}
            \hline
            Typen    & Java            & Typescript                           & Python                      & Golang     & Rust       & C          \\
            \hline
            Wert     & primitive Typen & primitive Typen                      & primitive Typen             & alle Typen & alle Typen & alle Typen \\
            \hline
            Referenz & Objekte         & Objekte & alles außer primitive Typen & Explizit & Explizit & Explizit \\
            \hline
        \end{tabular}
    \end{center}
    Bei den Sprachen Java, Typescript und Python werden nur primitive Typen als Werte und nicht als Referenzen in Variablen gespeichert.
    Deshalb wurden die primitiven Typen dieser Sprachen genauer verglichen:
    \begin{center}
        \begin{tabular}{| c || m{7em} | m{7em} | m{7em} | m{4.5em} | m{4.5em} | m{4.5em} |}
            \hline
            primitive Typen & Java & Typescript & Python \\
            \hline
            & byte, short, int, long, float, double, char, boolean
            & number, bigint, string, boolean
            & int, float, bool, str \\
            \hline
        \end{tabular}
    \end{center}
    Auffällig sind hierbei die Zusammenfassung der Typen byte, short, int, long in Java in den Typen int in Python, sowie die Zusammenfassung aller Zahlentypen, bis auf long, in number in Typescript.
    Java besitzt als einzige Sprache String nicht als primitiven Datentyp.

    \subsubsection{Analyse von Nullwerten}
    Nullwerte sind besonders aus Java bekannt und stellen das Fehlen eines Wertes dar.
    Es zählt zu der Definition eines Types dazu, zu definieren, ob der Typ Nullwerte erlaubt.
    Dies muss auch für Werte und Referenzen evaluiert werden.
    \begin{center}
        \begin{tabular}{| c || m{4em} | m{5em} | m{5em} | m{4.5em} | m{4.5em} | m{4.5em} |}
            \hline
            Nullwerte & Java & Typescript & Python & Golang & Rust     & C    \\
            \hline
            Wert      & nein & nein       & ja     & nein   & Explizit & nein \\
            \hline
            Referenz  & ja   & Explizit   & ja     & ja     & Explizit & ja   \\
            \hline
        \end{tabular}
    \end{center}
    Es ist klar zu erkennen, dass bis auf Python jede Sprache Wert-Variablen ohne Nullwerte darstellen kann.
    Referenzen können auch in jeder Sprache Nullwerte beinhalten.
    In Typescript und Rust muss dies bloß explizit definiert werden.
    Aus diesen Ergebnissen ergibt sich, dass die Unterteilung in Wert- und Referenz-Variablen auch die Unterteilung in Nullbare und nicht Nullbare Variablen abbildet.

    \subsubsection{Collectiontypen}
    Um 1:n- oder n:m-Beziehungen im Datenmodell modellieren zu können wurden drei Collection-Typen aus Java ausgewählt und passende Äquivalente zu finden.
    \begin{center}
        \begin{tabular}{| c || m{3em} | m{4.5em} | m{5.5em} | m{4.5em} | m{3em} | m{4.5em} |}
            \hline
            Collectiontypes & Java & Typescript & Python          & Golang     & Rust & C          \\
            \hline
            List            & ja   & ja (Array) & ja              & ja (slice) & ja   & ja (Array) \\
            \hline
            Set             & ja   & ja         & ja              & nein       & ja   & nein       \\
            \hline
            Map             & ja   & ja         & ja (dictionary) & ja         & ja   & nein       \\
            \hline

        \end{tabular}
    \end{center}
    Die gewählten Typen sind die am häufigsten verwendeten Collection-Typen.
    Eine Map beinhaltet eine 1:n-Beziehung und ermöglicht einen schnellen Zugriff.
    Eine Liste bildet eine n:m-Beziehung zwischen den modelliertem Element und dem Inhalt der Liste.
    Ein Set bildet eine n:m.Beziehung mit der Garantie, dass jeder enthaltener Wert einzigartig ist. \\

    In der Analyse der Liste gab es feine Unterschiede in der Implementierung.
    Typescript und C nutzen einen Array, jedoch verhält sich der Typescript Array wie eine Liste.
    In C sind Arrays in ihrer Größe bei ihrer Initialisierung festgelegt.
    Golang nutzt ein Konstrukt namens `slice'.
    Es kommt mit bestimmten Eigenschaften, kann jedoch für eine Liste genutzt werden. \\

    Ein Set findet sich nur in Golang und C nicht.
    Hier kann es durch eine Liste ersetzt werden.
    Die Garantien müssten selber verwaltet werden. \\

    Bei der Analyse der Map wurde nur in C keine Implementierung gefunden.
    Python nutzt für die Map den Namen `dictionary'.

    \subsection{Elemente eines Modells}
    Um mit dem \acrshort{dmf} Daten in Strukturen verschiedener Programmiersprachen darstellen zu können, müssen auch diese Abstrahiert werden.
    Dieser Abschnitt beschriebt wie aus den Analysen der Programmiersprachen die Abstraktion des \acrshort{dmf}s gebildet wurde.

    \subsubsection{Primitive Typen}
    Grundvoraussetzung sind die primitiven Typen und Referenzen zu anderen Elementen.
    Bei der Analyse wurde ein unterschiedliches Maß in der Feinheit der Zahlentypen festgestellt.
    Es gibt in \acrfull{sql} Datenbanksystem generell eine Unterscheidung zwischen ganzen Zahlen und rationalen Zahlen.
    Somit muss es eine Unterscheidung zwischen int und double geben.
    Es wird jedoch auch unterschieden wie groß ganze Zahlen werden, weshalb ein long Typ sinnvoll ist.
    Dieser kann auch mithilfe von bigint in Typescript abgebildet werden.
    Für die Verarbeitung von unbekannten Daten werden häufig Bytes genutzt.
    Von den drei verglichen Sprachen, beinhaltet nur Java den primitiven Typ.
    Die int-Typen der jeweiligen Sprachen ermöglichen jedoch ähnliche Operationen.
    Deshalb wurde auch Byte aufgenommen.
    Eine Unterscheidung zwischen float und double wurde nicht vorgenommen, da diese Unterscheidung in den Systemen, die die Typen enthalten, sehr wenig verwendet wird.\\
    String ist vor allem in Scriptsprachen ein primitiver Typ und wird auch von Datenbanken unterstützt.
    Deshalb wurde auch String als primitiver Typ ins \acrshort{dmf} aufgenommen.\\
    Im Gegensatz zu allen verglichenen Programmiersprachen besitzen \acrshort{sql}-Datenbanken Unterstützung für Datum- und Zeitstempel-Werte.
    Damit die Generation diese Werte in das Datenbankmodell übernehmen kann, wurden `date' und `datetime' als primitive Typen hinzugefügt. \\
    Abschließend gehört noch `boolean' zu den primitiven Typen.
    Wahrheitswerte werden sowohl in allen Programmiersprachen als auch in allen Datenbankensystem unterstützt. \\

    Somit beinhaltet das \acrshort{dmf} die folgenden primitiven Typen:

    \begin{center}
        \begin{tabular}{| c || c | c | c | c | c |}
            \hline
            Typ & ganze Zahlen & rationale Zahlen & Text & Zeit & Wahrheitswert \\
            \hline
            & byte, int, long & double & string & date, datetime & boolean \\
            \hline
        \end{tabular}
    \end{center}

    Diese primitiven Typen werden im \acrshort{dmf} in Argumenten abgebildet.
    Argumente bestehen aus einem primitiven Typen und einem Namen.
    Alle anderen Datentypen werden als Referenzen abgebildet.
    Vorgesehen ist nur Referenzen explizit als Nullbar generiert werden.

    \subsubsection{Funktionen}
    Funktionen gehören zu den Elementen, die sich in jeder Programmiersprache wiederfinden.
    Im \acrshort{dmf} werden Funktionen nur im Rückgabewert eingeschränkt.
    Statt mehreren Werten wie z.B.\ in Golang kann im \acrshort{dmf} nur ein einzelner Wert modelliert werden.
    Diese Einschränkung stammt aus vielen Sprachen, welche nur einen Wert unterstützen.

    \subsubsection{Komplexe Datentypen}
%    \begin{center}
%        \textit{Structured constructors. (+), types can be built up from these basic types by means of type. The type constructors in
%        our language include function spaces Cartesian products (x), record types (also called labeled Cartesian products), and variant types (also called labeled disjoint sums).\cite{https://dl.acm.org/doi/pdf/10.1145/6041.6042}}
%    \end{center}

    \begin{center}
        \textit{
            In nahezu allen Programmiersprachen gibt es die Möglichkeit, mit sogenannten zusammengesetzten oder komplexen Datentypen zu arbeiten. Ihnen ist gemeinsam, dass wir mehrere Werte nebeneinander dort abspeichern können.\cite{978-3-8348-9999-6.pdf}}
    \end{center}
    Das \acrshort{dmf} muss diese Datentypen auch abbilden können.
    Deshalb beinhaltet es Structs.
    Der Name wurde von der Programmiersprache C übernommen, da diese syntaktische Grundlage für fast alle Programmiersprachen dient. \\
    Im \acrshort{dmf} können Structs Argumente, Referenzen zu anderen Structs, Entities, Enums und Interfaces (siehe folgende Abschnitte) und Funktionen beinhalten.
    Funktionen gehören nicht zur Definition eines komplexen Datentyps, sondern stammen aus der Objekt-Orientierten-Programmierung.
    Da jedoch Funktionen auch ohne Objektorientierung für Datentypen generiert werden können, kann das \acrshort{dmf} diese Abstraktion unterstützen.
    \newline
    Für die Modellierung wird auch die Abstraktion von Datentypen essenziell sein.
    Dafür müssen Structs von anderen Structs erben und Funktionen von Interfaces implementieren können. \\
    Abstraktion funktioniert in jeder Sprache ein wenig unterschiedlich, weshalb das \acrshort{dmf} nur garantieren kann, dass die Variablen und Funktionen, die von einem Struct geerbt werden, im Generat vorhanden sind.
    Zum Beispiel in C könnte ein \acrshort{dmf} Generat keine Abstraktion generieren, sondern nur die Elemente kombinieren.

    \subsubsection{Identität einer Instanz in der Datenbank}
    Ein Modell im \acrshort{dmf} Framework soll in einer Datenbank gespeichert werden können.
    Dafür müssen Datenbankschlüssel definiert werden.
    Ein Schlüssel definiert die Identität einer Zeile in einer Tabelle.
    Diese Identität muss auch im Modell abgebildet werden.
    Das \acrshort{dmf} fügt deshalb den Typen `Entity' hinzu, welcher eine Identität besitzt.
    Er basiert auf dem Struct und kann somit Argumente, Referenzen und Funktionen beinhalten.
    Eine Entity muss die Definition eines Identifiers beinhalten. \\

    Die Vererbung bei Entities unterscheidet sich von Structs.
    Eine Entity darf sowohl von einem Struct als auch von einer Entity erben.
    Ein Struct darf nur von einem Struct erben.

    \subsubsection{Aufzählungen}
    Aufzählungen sind Bestandteil vieler Programmiersprachen.
    Häufig existieren sie als reine Liste aus Codesymbolen.
    Aus Sprachen wie Rust sind jedoch auch Aufzählungstypen, dessen Einträge konstante Werte beinhalten können, bekannt.
    %! suppress = MissingLabel
    \begin{lstlisting}[language=Rust, caption=Ein Enum in Rust von \cite{rustcookbook}]
enum IpAddr {
        V4(u8, u8, u8, u8),
        V6(String),
}
    \end{lstlisting}

    Diese Funktion kann auch in Sprachen dessen Enums diese Möglichkeit nicht beinhalten, durch Funktionen, die für den Enum-Eintrag den modellierten Wert zurückgeben, emuliert werden.

    Im \acrshort{dmf} lassen sich diese Werte mithilfe von Argumenten modellieren.
    Bei der Definition eines Enum-Eintrags müssen die Konstanten mit angeben werden.

    \subsubsection{Interfaces}
    Wichtig für die Abstraktion sind Interfaces.
    Sie stellen Funktionen bereit und können zusammen mit anderen Interfaces in Structs und Entities implementiert werden.

    \subsubsection{Organisation der Elemente}
    In großen Softwareprojekten werden Datentypen generell in Gruppen organisiert.
    Diese Gruppierung erfolgt meistens über das Dateisystem.
    Dabei repräsentiert ein Ordner eine Gruppe.
    Diese Gruppe wird meistens `Package' genannt. \\
    Das \acrshort{dmf} beinhaltet auch Packages.
    Diese werden jedoch nicht im Dateisystem modelliert, sondern sollen als Elemente im Modell enthalten sein.

    \subsection{Zuweisungen der Abstraktionen}

    Damit diese Abstraktion genutzt werden kann, müssen für jeden abstrakten Typen im \acrshort{dmf} eine Zuweisung in jeder Sprache festgelegt werden.

    \subsubsection{Java}
    \begin{center}
        \begin{tabular}{| c || c |}
            \hline
            Element   & Java           \\
            \hhline{|=||=|}
            package   & Java Package   \\
            \hline
            struct    & Java Klasse    \\
            \hline
            entity    & Java Klasse    \\
            \hline
            interface & Java Interface \\
            \hline
            enum      & Java Enum      \\
            \hline
        \end{tabular}
    \end{center}
    Die \acrshort{dmf} Elemente können sehr gut in Java übersetzt werden.
    Für die Entity kann sogar die Identität mithilfe der Implementation von der Methoden `hashCode' und `equals' übernommen werden.
    Die Enums unterstützen auch die zusätzlichen Argumente.

    \begin{center}
        \begin{tabular}{| c || c |}
            \hline
            Datentyp & Java                    \\
            \hhline{|=||=|}
            byte     & byte                    \\
            \hline
            int      & int                     \\
            \hline
            long     & long                    \\
            \hline
            double   & double                  \\
            \hline
            boolean  & boolean                 \\
            \hline
            string   & java.lang.String        \\
            \hline
            date     & java.time.LocalDate     \\
            \hline
            datetime & java.time.LocalDateTime \\
            \hline
        \end{tabular}
    \end{center}
    Das \acrshort{dmf} kann bei den Zahlen und Wahrheitswerten genau auf Java übersetzt werden.
    Für Text- und Zeitwerte werden Klassen der Standardbibliothek verwendet.

    \subsubsection{Typescript}
    \begin{center}
        \begin{tabular}{| c || c |}
            \hline
            Element   & Typescript           \\
            \hhline{|=||=|}
            package   & Ordner               \\
            \hline
            struct    & Typescript Klasse    \\
            \hline
            entity    & Typescript Klasse    \\
            \hline
            interface & Typescript Interface \\
            \hline
            enum      & Typescript Enum      \\
            \hline
        \end{tabular}
    \end{center}
    TODO Nach Entwicklung des Typescript Generators

    \begin{center}
        \begin{tabular}{| c || c |}
            \hline
            Datentyp & Typescript \\
            \hhline{|=||=|}
            byte     & number     \\
            \hline
            int      & number     \\
            \hline
            long     & bigint     \\
            \hline
            double   & number     \\
            \hline
            boolean  & boolean    \\
            \hline
            string   & string     \\
            \hline
            date     & Date       \\
            \hline
            datetime & Date       \\
            \hline
        \end{tabular}
    \end{center}
    TODO Nach Entwicklung des Typescript Generators

\end{document}