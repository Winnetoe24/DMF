\documentclass[./einleitung.tex]{subfiles}
\normalsize

\begin{document}
    \section{Semantische Verarbeitung des \acrfull{ast}}\label{sec:semantische-verarbeitung-des-ast}
    Mithilfe von Treesitter kann nun ein Parser für die vorher beschriebene Grammatik generiert werden.
    Dieser Parser ist in der Lage, einen \acrshort{ast} aus einem Text zu generieren.
    Damit der Generator (siehe~Kapitel~\ref{sec:der-generator}) und der \acrshort{lsp}-Server (siehe~Kapitel~\ref{sec:der-lsp-server}) die Informationen des \acrshort{ast} effektiv verarbeiten können,
    muss der \acrshort{ast} vorher verarbeitet werden.
    Während dieser Verarbeitung wird zuerst ein semantisches Modell erstellt.
    Anhand dieses semantischen Modells werden semantische Regeln überprüft.\\

    Diese Softwarekomponente ist die erste Komponente welche mithilfe von Golang implementiert wird.

    \subsection{Das semantische Modell}\label{subsec:das-semantische-modell}
    Das semantische Modell bildet alle Informationen, die aus dem \acrshort{ast} entnommen werden können, ab.
    Dazu gehören Referenzen zum \acrshort{ast} für Positionen in der Modelldatei, die verschiedenen PackageElemente und die NamedElemente.
%  Wie sollte hier adas Modell beschrieben werden?
%      Wäre ein UML Modell richtig?

    \subsubsection{ErrorElemente}
    Für eine annehme und effiziente Entwicklung ist die verständliche Kommunikation von Fehlern essenziell.
    Deshalb enthält das semantische Modell das ErrorElement.
    \begin{lstlisting}[language=go]
package err_element
type ErrorElement struct {
	Fehler FehlerStelle
	Error  error
	Cause    *FehlerStelle
	rendered *string
}

type FehlerStelle struct {
	ContextNode *tree_sitter.Node
	ModelCode   string
	Node        *tree_sitter.Node
}
    \end{lstlisting}
    Mithilfe dieser Strukturen lassen sich Fehler festhalten, ohne die spätere Darstellung mit einzubeziehen zu müssen.
    Der Error dokumentiert um welchen Fehler es sich handelt. \\
    Die FehlerStelle `Fehler' dokumentiert den Modellcode, der falsch ist. \\
    Die FehlerStelle `Cause' dokumentiert den Modellcode, wodurch der Modellcode, welcher in der `Fehler'-FehlerStelle enthalten ist, falsch wurde. \\
    Jede FehlerStelle beinhaltet zwei Nodes.
    Diese beinhalten die Positionen in der Modelldatei.
    Die ContextNode ist optional und wird gesetzt wenn für einen Fehler der umliegende Code wichtig.
    Es handelt sich dabei um eine der Parent-Nodes. \\
    Der ModelCode wird erst beim Rendern der FehlerStelle genutzt.
    Die Variable dient als Zwischenspeicher für den Code den die Nodes referenzieren. \\ \\
    ErrorElemente werden in der gesamten Semantik-Komponente genutzt und erzeugt.
    Spätere Komponenten nutzen die ErrorElemente, um den Entwickler*innen die Fehler zu erläutern.


    \subsection{Übersetzen des \acrshort{ast}s}\label{subsec:ubersetzen-des-asts}
    Die Übersetzung des \acrshort{ast}s beginnt mit dem Erstellen eines SemanticContext.
    Dieser Kontext beinhaltet die erkannten Fehler, das bisherige Modell, den Text der Modelldatei und die TreeCursor.
    Mithilfe des TreeCursors kann der \acrshort{ast} durchlaufen werden.
    Der Kontext durchläuft den \acrshort{ast} in der PreOrder-Reihenfolge.
    Für jedes Element des \acrshort{ast}s enthält der Kontext eine Methode zum Parsen in der die entsprechenden Elemente aus dem Semantik-Modell erzeugt werden.\\
%    Beispiel Call Tree
    Sollte der \acrshort{ast} Import Statements beinhalten, so werden zunächst die referenzierten Modelle verarbeitet.
    Die importierten Packages werden nun in das Modell übernommen.
    Das Laden der verschiedenen Modelle wird mithilfe einer Callbackstruktur außerhalb der Semantik-Komponente definiert.
    So können verschiedene Logiken genutzt werden. \\
    Der Callback-Aufruf beinhaltet den kompletten Parse-Durchlauf.
    Im \acrshort{lsp}-Server werden die Ergebnisse zwischengespeichert, sodass die Verarbeitung von Dokumenten die gerade geöffnet sind beschleunigt wird.\\
    In der Kontext-Struktur des Parsens wird ein Lookup (siehe~\ref{subsubsec:lookup})) genutzt um die importierten PackageElemente zwischenzuspeichern.
    Während des Parsen von erweiternden Elementen wird der Lookup referenziert und die Eigentschaften in neue Element
    Am Ende des Parsen ist Modell vollständig mit allen gültigen Elementen.
    Elemente die durch den Parser im \acrshort{ast} als Fehlerhaft gekennzeichnet wurden, werden ignoriert.

    \subsubsection{Der Lookup}\label{subsubsec:lookup}
    Um anhand des vollständigen Namens (Package Pfad + Name) ein Element schnell zu finden, wird nach dem semantischen Parsen ein Lookup erstellt.
    Dieser Lookup nutzt eine Map für den schnellen Zugriff.
    Zum Befüllen des Lookups wird das semantische Modell durchlaufen.
    Jedes erreichte Package Element wird dem Lookup hinzugefügt.
    Beim Packages werden auch alle enthaltenen Elemente durchlaufen.

    \subsection{Das Datenbankmodell}\label{subsec:das-datenbankmodell}
    Um das semantische Modell auf ein Datenbankschema abzubilden, muss es eine zentrale Übersetzung geben.
    In dieser wird aus dem semantischen Modell Tabellen und Spalten generiert.
    \begin{lstlisting}[language=Go, captio=Datenstruktur für Tabellen und Spalten, label=lst:tableColumn]
type Table struct {
	Name string
	Columns           []*Column
	TablesForElements []TableReference
}

type Column struct {
	Name       string
	Type       base.PrimitivType
	PrimaryKey bool
	ForeignKey *ColumnReference
	Kommentar  string
}

type ColumnReference struct {
	TableName string
	Column    *Column
}

type TableReference struct {
	Referenzname string
	IsExtra bool
	*Table
}
    \end{lstlisting}
    Bei der Generierung werden zunächst die \textbf{Tabellen für die Entitäten} berechnet.
    Hierbei werden die Elemente in einer Entität in Spalten übersetzt.
    Dafür wird ein Zwischenmodell genutzt (siehe~\nameref{ch:anhang}).\\
    Bei der Übersetzung wird ein Argument zu einer Spalte mit dem gleichen Datentyp.
    Bei Referenzen wird nach dem referenzierten Typ unterschieden.\\
    Wird ein \textbf{Struct} Referenziert, werden die Spalten der Elemente des Structs zur Tabelle hinzugefügt.\\
    Wird eine \textbf{Entity} Referenziert, werden die Spalten des Identifiers der Entität hinzugefügt.
    Bei diesen Spalten wird der `ForeignKey' mit einer Referenz zu referenzierten Spalte und mit dem Tabellennamen befüllt.\\
    Wird ein \textbf{Enum} referenziert, so wird nur in einer Spalte der Index der Enum-Konstante gespeichert.\\
    Werden \textbf{Multireferenzen} verwendet, so werden keine Spalten hinzugefügt.
    Während der Generation wird im Zwischenmodell (siehe~\nameref{ch:anhang}) die Auslagerung dieser Referenzen vermerkt.
    So können zusätzliche Tabellen in einem späteren Schritt generiert werden.
    Diese zusätzlichen Tabellen werden in der Variable `TablesForElements' gespeichert.\\
    Auch bei Referenzen zu \textbf{Interfaces} wird eine ausgelagerte Tabelle generiert.
    Jedoch handelt es sich hierbei um eine Tabelle, welche von verschiedenen Entitäten genutzt werden kann.
    Deshalb wird dies in der Tabellenreferenz in der Variable `isExtra' festgehalten.
    Tabellenreferenzen mit dem `isExtra' Wert werden nur einmal bei der \acrshort{sql}-Schema-Generation generiert.
    \\\\
    Nach der Generation der Tabellen für die Entitäten werden die \textbf{zusätzlichen Tabellen} generiert.
    \textbf{Tabellen für Multireferenzen} enthalten immer den Primärschlüssel der Entität, welche die Referenz enthält.
    So können Zeilen in der Tabelle zu einer Spalte in der Entität-Tabelle zugeordnet werden.\\
    Der Typ der Multireferenz entscheidet über die weiteren Spalten.
    Für die generischen Typen wird die Übersetzung aus dem ersten Schritt übernommen.\\
    Eine \textbf{Map} generiert den Typ des Keys als Spalten mit dem Prefix `Key'.
    Diese Spalten sind auch Teil des Primärschlüssels.
    Der Typ des Values der Map (2. generischer Typ) wird mit dem Prefix `Value' generiert.\\
    \textbf{Listen} enthalten eine Spalte namens `Index' mit dem Typ `int' als Teil des Primärschlüssels.
    Diese Spalte repräsentiert den Index in der Liste.
    Die Spalten des Inhalt-Typs der Liste werden mit dem Präfix `Value' generiert.\\
    Wie bei einer Liste werden die Spalten des Inhalt-Typs eines \textbf{Sets} mit dem Präfix `Value' generiert.
    Jedoch enthält ein Set keine Index-Spalte, sondern nutzt die Spalten des Inhalts als Teil des Primärschlüssels.\\
    \textbf{Tabellen für Interfaces} enthalten einen Index und die Referenzspalten von jedem Typ welcher das Interface implementiert.
    Der Index ist der Primärschlüssel und wird in anderen Tabellen referenziert.
    \\\\
    Nachdem alle Tabellen und ausgelagerten Tabellen generiert wurden, werden die Tabellen in einem Schema gespeichert.
    Dieses Schema wird später vom Generator (siehe~\ref{sec:der-generator}) genutzt.
    Im Schema wird neben den Tabellen auch die Reihenfolge der Entitäten, dessen Tabellen generiert wurden, in der Modelldatei gespeichert.
    So können die Tabellen immer in der gleichen Reihenfolge im \acrshort{sql}-Schema generiert werden.
    Dies ist wichtig um Unterschiede zu identifizieren.
    \begin{lstlisting}[language=Go, caption=Datenstruktur Schema, label=lst:schema]
type Schema struct {
	TableLookUp map[string]*Table
    FileOrdner []string
}
    \end{lstlisting}
    \subsection{Die semantischen Regeln}\label{subsec:die-semantischen-regeln}
    Die semantischen Regeln basieren alle auf dem Typen `walkRule' und verarbeiten das vorher erstellte semantische Modell.
    Dieser Typ verallgemeinert das Iterieren über den Lookup und nutzt eine Instanz der `iWalkRule' um die Elemente zu verarbeiten.
    \begin{lstlisting}[language=go]
package semantic_rules
type walkRule struct {
	lookup    *smodel.TypeLookUp
	elements  []errElement.ErrorElement
	iWalkRule iWalkRule
}
    \end{lstlisting}
    Semantische Regeln implementieren das Interface `iWalkRule' und erweitern die `walkRule'.
    Durch das Nutzen der eignen Instanz werden die Methoden der Regel-Implementierung aufgerufen, um Elemente zu verarbeiten.
    \begin{lstlisting}[language=go]
package semantic_rules
func newComputeSuperTypes(lookup *smodel.TypeLookUp) *computeSuperTypes {
	types := computeSuperTypes{
		walkRule: &walkRule{
			lookup:   lookup,
			elements: make([]errElement.ErrorElement, 0),
		},
	}
    types.iWalkRule = &types
    return &types
}\end{lstlisting}

    Die Reihenfolge der semantischen Regeln ist sehr relevant, denn sie überprüfen nicht nur das semantische Modell, sondern setzen auch Referenzen und befüllen Lookups innerhalb der Elemente. \\
    Im Folgenden werden die Regeln in ihrer Ausführungsreihenfolge erläutert.

    \subsubsection{Compute Supertypes Regel}
    Die ``Compute Supertypes Regel'' ermittelt und überprüft die Supertypen der PackageElementen.
    Dazu gehören alle Referenzen in den Extends- und Implements-Blöcken.
    Dabei werden folgende Bedingungen überprüft:
    \begin{enumerate}
        \item Der referenzierte Typ muss im Lookup vorhanden sein.
        \item Die Vererbung darf nicht rekursiv sein.
        \item Structs dürfen nur von Structs erben.
        \item Entities dürfen nur von Structs und Entities erben.
        \item Es können nur Interfaces implementiert werden.
        \item Ein Interface kann sich nicht selber implementieren.
    \end{enumerate}

    Nachdem die referenzierten Type ermittelt wurden, werden Referenzen zu diesen Typen in die jeweiligen PackageElemente eingetragen.
    So kann später noch schneller auf diese Elemente zugegriffen werden.

    \subsubsection{Compute Elements Regel}
    Die ``Compute Elements Regel'' ermittelt alle Elemente innerhalb der PackageElemente.
    Diese werden als NamedElements in dem jeweiligen PackageElement eingetragen, sodass mithilfe des Namens schnell auf das Element zugegriffen werden kann.
    Zu den NamedElements gehören:
    \begin{enumerate}
        \item Argumente
        \item Referenzen
        \item Multireferenzen
        \item Funktionen
        \item Konstanten
    \end{enumerate}
    Beim Hinzufügen eines Elementes wird überprüft, ob der Name schon von einem Element in der Map genutzt wird. \\ \\
    Die `Compute Elements Regel' einzigartig, denn sie überschreibt das Verhalten des iterierens durch den Lookup mit einem zweifachen Durchlauf.
    Denn um alle geerbten Elemente zu Bestimmen müssen die NamedElements-Maps der anderen PackageElemente schon die eigenen Elemente enthalten.
    \begin{lstlisting}[language=Go]
package semantic_rules
func (c *computeElements) walk() []errElement.ErrorElement {
	c.walkRule.walk()

	for _, element := range *c.lookup {
		c.handleAbstraction(element)
	}

	return c.elements
}
    \end{lstlisting}
    Das Bestimmen der gerbten Elemente (handleAbstraction) ist rekursiv implementiert.
    So werden auch alle Elemente, die das erweiterte Element erbt mit eingeschlossen.

    \subsubsection{Check Entity Identifier Regel}
    Die `Check Entity Identifier Regel' überprüft, ob alle im Entity-Identifier referenzierten Variablen enthalten sind.

    \subsubsection{Check Enum Constants Regel}
    Die Konstanten eines Enums müssen semantische Regeln folgen.
    Deshalb überprüft diese Regel folgende Bedingungen für jede Konstante:
    \begin{enumerate}
        \item Der Name jeder Konstante muss innerhalb eines Enums einzigartig sein.
        \item Der erste Wert muss ein Integer sein (Index).
        \item Die Anzahl der Werte muss mit der Anzahl der Argumente im Enum übereinstimmen.
        \item Der Typ der Werte muss mit den entsprechenden Argumenten übereinstimmen. \\
        Für die Reihenfolge der Argumente wird die Reihenfolge in der Modelldatei genutzt.
    \end{enumerate}

    \subsubsection{Check Referenzen Regel}
    Die `Check Referenzen Regel' überprüft, dass alle Referenzen existieren.
    In folgenden Elementen werden Referenzen überprüft:
    \begin{enumerate}
        \item Referenzen in Structs und Entities.
        \item Return Typ und Parameter von Funktionen in Structs, Entities und Interfaces.
        \item Generische Typen der Multireferenzen in Structs und Entities.
    \end{enumerate}

\end{document}