\chapter{Bewertung von Abschlussarbeiten}

An der Fakultät für Informatik wird derzeit keine schematisierte Bewertung eingesetzt,
sondern die Gutachterinnen und Gutachter bewerten die Arbeit in einem Gutachten.
Dabei können sie individueller auf die etwaigen Besonderheiten der Arbeiten eingehen
(bspw.{} Probleme bei den Themenstellungen berücksichtigen).
Aber auch je nach Typ der Arbeit können hier unterschiedliche Gewichtungen entstehen.
Bei theoretischen Arbeiten, empirischen Arbeiten, oder Literaturarbeiten werden unterschiedliche
Gewichtungen der einzelnen Aspekte notwendig sein.

Im Folgenden finden Sie eine Übersicht über typische Faktoren, die die Bewertung beeinflussen.
Eine ausführlichere Diskussion der Anforderungen und Bewertungskriterien finden Sie
beispielsweise bei \textcite{Voss2022}, \textcite{Deininger2017} oder \textcite{Boles21}.

\paragraph*{Eigeninitiative und Selbständigkeit}
Arbeiten, die von Studierenden in hoher Autonomie vorangetrieben werden, werden in der Regel
besser bewertet als Arbeiten bei denen eine große Unterstützung durch die Betreuenden notwendig ist.
Keinesfalls sollten Sie aber auf regelmäßige Treffen verzichten! Hiermit ist gemeint dass sie proaktiv
die Arbeit voranbringen, und die Betreuenden werden sich freuen wenn Sie jede Woche Fortschritte berichten.
Dies beginnt oft schon bei der Themenfindung -- viele sehr gute Arbeiten beginnen mit eigenen Ideen
der Studierenden.

Eine Arbeit, die mit Ach und Krach nur die Mindestanforderungen bei einem einfachen erreicht, wird natürlich anders
bewertet werden als eine Arbeit die über die gesetzten Ziele hinausgegangen ist, oder ein besonders schweres Thema in Angriff genommen hat.

\paragraph*{Systematische und wissenschaftliche Arbeitsweise}
Für eine wissenschaftliche Arbeit sollen Sie zielgerichtet vorgehen, beispielsweise Forschungsfragen stellen,
diese bearbeiten, und anschließend mit geeigneten Experimente die Ergebnisse kritisch validieren.
Auch in ihrer Ausarbeitung sollte eine klare Struktur erkennbar sein, die das wissenschaftliche Vorgehen
nachvollziehbar macht. Experimente sind geeignet automatisiert, so dass sie leicht repliziert und auf neue
Datensätze angewendet werden können.

Vielen hilft es dabei, bereits beim Exposé Forschungsfragen zu formulieren und geeignete Experimente zu
konzipieren diese zu verifizieren oder zu falsifizieren. Auch ein Zeitplan -- mit ausreichendem Puffer
am Ende der Bearbeitungszeit -- ist dabei hilfreich.
Die Betreuenden helfen Ihnen dabei sicher gerne.

\paragraph*{Recherche und Dokumentation}
Es soll nachvollziehbar sein, dass Sie die Aufgabenstellung verstanden, und nötige Vorarbeiten recherchiert haben.
In der Arbeit soll dokumentiert sein, wie Sie zu ihren Schlüssen gekommen sind. Dazu werden die nötigen
Schritte logisch und verständlich, aber dennoch prägnant und zielführend präsentiert, gegebenenfalls durch
geeignete Beispiele und Abbildungen unterstützt.
In der Regel werden Sie dazu weitere relevante Literatur identifizieren müssen.
Bei einer Arbeit deren Fokus auf Literatur liegt, sind hier die Ansprüche natürlich wesentlich höher,
als bei einer Arbeit bei der die Literatur weitgehend vorgegeben ist!

\paragraph*{Wissenschaftliche Form}
Aussagen sind durch korrekte Quellenangaben belegt, dabei ist die Herkunft der Ideen anhand der Quellenangaben
und des Literaturverzeichnisses leicht nachvollziehbar (\enquote{lege artis}, nach den \enquote{Regeln der Kunst} der Informatik).
Bei Bedarf werden Aussagen angemessen kritisch hinterfragt, diskutiert, und verifiziert bevor sie übernommen werden.
Fachbegriffe (bspw.{} effizient, effektiv, signifikant, skalierbar) werden angemessen eingesetzt und ggf.{} belegt. 
Der Schreibstil und die Sprache entsprechen den Anforderungen einer wissenschaftlichen Arbeit, die Ausarbeitung ist sauber und gut lesbar.
Die Lesenden werden mitgenommen und motiviert den Ausführungen der Arbeit zu folgen.

Negativ hingegen fallen Arbeiten auf, bei denen beispielsweise die Einleitung
im auf Hochglanz polierten und mit Superlativen gespickten ChatGPT-Stil geschrieben ist (\enquote{meticulously}, \enquote{intricate}, und \enquote{commendable} sind beispielsweise Wörter die ChatGPT übermäßig verwendet),\footnote{Zum Einsatz von ChatGPT beachten Sie auch die Eidesstattliche Versicherung, die sie unterzeichnen müssen, und den Abschnitt \enquote{KI-Schreibwerkzeuge und wissenschaftliches Fehlverhalten} im Rechtsgutachten von \textcite{10.13154/294-9734} sowie die Handreichung zu ChatGPT der TU Dortmund, \url{https://digitale-lehre.tu-dortmund.de/tools/chatgpt/}.}
der Hauptteil dafür dann voller Rechtschreib- und Grammatikfehler ist, und keine zwei Formeln die gleiche Notation verwenden.
Viele Quellen sind Internetquellen fragwürdiger Qualität, und es ist nicht klar welche Aussagen damit belegt werden sollen,
und ob die Quellen diese Aussagen überhaupt untermauern können.

\medskip
Schlussendlich wird aber jede Arbeit durch die Gutachter:innen individuell gewürdigt, und diese können hier auf individuelle Besonderheiten Rücksicht nehmen.