\chapter{Themenfindung und Anmeldung}

Aktuelle und genauere Informationen:
\url{https://cs.tu-dortmund.de/studium/im-studium/weitere-lehrveranstaltungen/abschlussarbeiten/}

\section{Themen für Abschlussarbeiten}

Abschlussarbeiten werden von einem/einer Professor*in, einem/einer
Juniorprofessor*in oder einem habilitierten Mitglied der Fakultät für
Informatik ausgegeben und betreut.

Themenvorschläge finden Sie auf den Webseiten der jeweiligen Arbeitsgruppen.

Sie können \emph{eigene Themen vorschlagen}, auch mit externen Partnern.
Sprechen Sie dazu thematisch passende Hochschullehrer*innen an.
Ob ein Thema die Anforderungen erfüllt entscheidet der/die jeweilige Hochschullehrer*in.

Orientieren Sie sich an den Themenvorschlägen und den Themen abgeschlossener Arbeiten,
um einen Eindruck zu bekommen wie hier die Erwartungen sind.

Für Abschlussarbeiten in Industriekooperationen ist ggf.{} ein längerer Vorlauf nötig,
insbesondere wenn es um non-disclosure agreements (NDAs) und ähnliches geht,
die von der Rechtsabteilung geprüft werden müssen.
Für solche Arbeiten ist es allgemein empfehlenswert wenn sich zunächst die Industriepartner
mit einem/einer Hochschullehrer*in verständigen, \emph{bevor} das Thema Studierenden angeboten wird,
um keine unnötigen Wartezeiten zu verursachen.

\section{Exposé}

Es hat sich bewährt dass noch \emph{vor der Anmeldung} ein Exposé erstellt wird.

Dieses kurze Dokument (ca.{} 2-4 Seiten) dient dazu, die Aufgabenstellung
genauer festzulegen und zu dokumentieren, damit klar ist dass der/die Bearbeiter*in des Themas
dieses genauso verstanden hat wie der/die Aufgabensteller*in.

Überlegen Sie sich in diesem Zuge auch, ob sie auf Englisch oder auf Deutsch schreiben wollen.
Bedenken Sie, dass auf Deutsch schnell ein ziemlicher Kauderwelsch entsteht, da
die Fachbegriffe auf Englisch sind, und es oft keine gängige Übersetzung gibt.
Zudem fällt es den meisten leichter in Englisch einen einfachen Satzbau zu verwenden,
während Sie in Deutsch zu unnötig komplizierten \enquote{Schachtelsätzen} neigen.

\pagebreak[2]
Das Exposé sollte umfassen:
\begin{itemize}
\item
Vorläufiger Titel der Arbeit
\item
Problemstellung und kurzer Stand der Forschung
\item
Forschungsfragen, die in der Arbeit beantwortet werden sollen und Konzept für einen Lösungsansatz der untersucht werden soll, Methodik
\item
ggf.{} Plan auf welchen Daten und wie die Ergebnisse evaluiert werden sollen
\item
Vorgehensweise und Zeitplan
\item
Startliteratur
\end{itemize}

Oft können Teile des Exposé später für die Ausarbeitung weiterverwendet werden.

Forschungsfragen sollen idealerweise \textbf{spezifisch} (konkret), \textbf{präzise}, \textbf{komplex} (nicht nur ja/nein),
\textbf{machbar}, \textbf{verifizierbar oder falsifizierbar} und \textbf{relevant} für die Informatik sein.

Bei der Evaluation achten Sie insbesondere auf die Wahl von geeigneten
\enquote{baselines}, einfachen Vergleichsverfahren mit denen Sie Referenzwerte
etablieren können um die Ergebnisse besser einschätzen zu können.

Besprechen Sie das Exposé mit ihren Betreuer*innen!

\section{Zeitplan}

Ihr Exposé sollte auch einen Zeitplan beinhalten.
Das Zeitfenster für eine Bachelorarbeit beträgt 4 Monate (17 Wochen).
Ein möglicher Zeitplan für eine empirische Arbeit ist in Tabelle~\ref{tab:Zeitplan} dargestellt.
Bei einer Masterarbeit verlängern sich die Zeiträume entsprechend.

Planen Sie genug Puffer ein für Fehlersuche und um anschließend alle Experimente erneut starten zu können...

\begin{table}[bt]
\centering
\caption{Zeitplanvorschlag für eine empirische Arbeit}\label{tab:Zeitplan}
\begin{tabular}{@{}cp{60mm}p{70mm}@{}}
\toprule
\textbf{Wo.} & \textbf{Aufgabe} & \textbf{Textarbeit} \\
\midrule
 1 & Literaturrecherche                & Verwandte Arbeiten schreiben \\
 2 & Literaturrecherche,\newline Programmieren & Verwandte Arbeiten schreiben \\
 3 & Programmieren                     & Verwandte Arbeiten schreiben \\
 4 & Programmieren                     & Hauptteil schreiben \\
 5 & Programmieren                     & Hauptteil schreiben \\
 6 & Programmieren, Debugging          & Hauptteil schreiben \\
 7 & Debugging, Programmieren          & Hauptteil schreiben \\
 8 & Debugging, (Programmieren)        & Hauptteil schreiben \\
\midrule
 9 & Evaluation, Debugging             & Verwandte Arbeiten, Hauptteil \\
10 & Evaluation, Debugging             & Verwandte Arbeiten, Auswertung \\
11 & Evaluation, Plots                 & Auswertung schreiben\\
12 & Evaluation, Plots,\newline Evaluation überprüfen & Auswertung schreiben\\
13 & Evaluation, Plots                 & Einleitung und Schlussfolgerungen\\
14 & Feinschliff Plots                 & Text zur Grammatik und Rechtschreibkontrolle an Freunde, Familie, etc. \\
15 & Quellen überprüfen                & Korrekturen und Feinschliff\\
16 & Fertig                            & Korrekturen und Feinschliff\\
\midrule
17 &  Fertig & Puffer \\
\bottomrule
\end{tabular}
\end{table}

\section{Prozess}

Die Anmeldung einer Arbeit erfolgt (Stand Anfang 2024) in der Regel wie folgt:

\begin{enumerate}
\item
Der/die Zweitgutachter*in stellt den \textbf{Antrag auf Zweitbegutachtung} (mit Name des/der Student*in sowie Titel).
\item
Der Prüfungsausschuss (bei Standardfällen kommissarisch durch den/die Vorsitzende*n) bestimmt die Gutachter*innen.
\item
Die zentrale Prüfungsverwaltung prüft die Zulassungsvoraussetzungen (Mindestzahl an ECTS, etc.) und erstellt den \textbf{Laufzettel}.
\item
Der/die Erstgutachter*in unterzeichnet den Laufzettel, damit beginnt die Bearbeitungszeit.
\item
Gegen Ende der Bearbeitungszeit, auch nach Abgabe, halten Sie einen Abschlussvortrag im Oberseminar der Arbeitsgruppe.
\item
Die Abgabe erfolgt in Exabase. Hier laden Sie die PDF ihrer Arbeit, die Eidesstattliche Versicherung,
und möglichst ein Archiv mit den relevanten Daten ihrer Arbeit (bspw.{} Quellcode, Protokolle, Skripte -- alles was für die Reproduzierbarkeit nötig ist) hoch.
\item
Die Gutachter*innen bewerten die Arbeit und laden ein Gutachten in Exabase hoch.
\item
Wenn die Gutachten vorliegen \emph{und} Sie ihren Abschlussvortrag gehalten haben, ist die Prüfungsleistung abgeschlossen.
\end{enumerate}
