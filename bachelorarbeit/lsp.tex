\documentclass[./einleitung.tex]{subfiles}
\usepackage{mdwtab}
\normalsize

\begin{document}
    \section{Der \acrshort{lsp}-Server}\label{sec:der-lsp-server}
    Der \acrshort{lsp}-Server besteht aus 3 Bereichen:
    \begin{enumerate}
        \item Dem Modell für das \acrlong{lsp}.
        \item Der Implementierung der Server-Schicht, welche die Kommunikation mit den Clients verwaltet und die Services aufruft.
        \item Die Service-Schicht, welche Funktionen des \acrshort{lsp}s in Services implementiert.
    \end{enumerate}

    In den folgenden Abschnitten werden der Aufbau, die Aufrufe und die Funktionen dieser Schichten erläutert.
    \subsection{Das \acrfull{lsp}}\label{subsec:das-acrfull{lsp}}\hypertarget{lsp-tech}
    % LSP Protokoll ist modular
    % Verschiedene Packages aufzählen
    TODO
    \subsection{Die Server Implementierung}
    % Wie funktioniert JSON-RPC
    % Wie parsed Golang json
    % Wie werden Service Aufgerufen

    \subsubsection{Abstraktion der Server-Client-Verbindung}
    Da das \acrshort{lsp} Medium unabhängig ist, muss der Server eine Abstraktion für die Verbindung bereitstellen.
    \begin{lstlisting}[language=Go]
package connect
type Connection interface {
	// WriteMessage Writes Message. May queue the Message the syncronise the Writing
	WriteMessage(message protokoll.Message)

	// WaitForMessage Waits for the next Message and returns the Message or the error.
	// Call to this method blocks execution.
	WaitForMessage() (protokoll.Message, error)

	BlockResponse(id json.RawMessage)
	Close() error
}
    \end{lstlisting}
    Das Interface Connection stellt diese Abstraktion bereit.
    Es wird für jeden Verbindungstyp implementiert.

    % Wie funktioniert die Kommunikation über HTTP
    \subsection{Die \acrshort{lsp}-Services}
    Die Schnittstelle für alle Services zu der Server-Schicht bildet das MethodHandler-Interface:
    \begin{lstlisting}[language=Go]
package service
type MethodHandler interface {
	Initialize(params *initialize.InitializeParams,
                result *initialize.InitializeResult)
	GetMethods() []string
	HandleMethod(message protokoll.Message)
}
    \end{lstlisting}
    Jeder Service implementiert die drei Methoden.
    \begin{enumerate}
        \item \textbf{Initialize} \\
        Der Service liest die Fähigkeiten des Clients und konfiguriert sich selbständig.
        Sollte der Client den Service nicht unterstützen, muss er sich deaktivieren.
        Der Service schreibt seine Fähigkeiten in die Antwort des Servers.
        \item \textbf{GetMethods} \\
        Gibt die Methoden aus dem \acrshort{lsp} zurück für die der Service Meldungen verarbeitet.
        \item \textbf{HandleMethod} \\
        Verarbeite die Nachricht.
    \end{enumerate}


    \subsubsection{FileService}
    Der FileService ist die Schnittstelle zwischen den Dateien, den Parsern und den restlichen Services.
    Wenn ein anderer Service auf Dateien, semantische Modell oder den Lookup zugreifen möchte, werden Methoden des FileServices genutzt.
    \begin{lstlisting}[language=Go]
package fileService
type FileService struct {
	handleMap map[string]*fileHandle
	listeners []FileChangeListener
	con       connect.Connection
}
    \end{lstlisting}
    In der ``handleMap'' werden ``fileHandles'' gespeichert.
    Ein FileHandle speichert alle Daten zu einer Datei und wird mit jeder Veränderung aktualisiert.
    \begin{lstlisting}[language=Go]
package fileService
type fileHandle struct {
	FileContent string
	Ast         *tree_sitter.Tree
	Model       *smodel.Model
	LookUp      *smodel.TypeLookUp
	Version     int32
}
    \end{lstlisting}
    Um einen FileHandle zu erzeugen, wird der Inhalt der Datei mithilfe Semantik-Schicht geparst.\\
    Wird der Dateiinhalt geändert, so werden die Änderungen an die Semantik-Schicht übergeben.
    Dort werden die Änderungen zum iterativen Parsen des neuen Dateiinhalts genutzt.
    Abschließend werden der Lookup erzeugt und die semantischen Regeln durchlaufen. \\
    \paragraph{FileChangeListener}
    Es gibt Funktionen im \acrshort{lsp} die nicht durch eingehende Nachrichten ausgelöst werden, sondern nach Dateiänderungen automatisch an den Client übermittelt werden.
    Dafür gibt es im \acrshort{dmf}-\acrshort{lsp}-Server die FileChangeListener.
    \begin{lstlisting}[language=Go]
package fileService
type FileChangeListener interface {
	// HandleFileChange gets called when the FileService
    // finishes parsing the File.
    // It may be called in its own routine.
    // Changes to the Parameters are ignored.
	HandleFileChange(file protokoll.DocumentURI, fileContent string,
                    ast *tree_sitter.Tree, model *smodel.Model,
                    lookup smodel.TypeLookUp,
                    errorElements []errElement.ErrorElement,
                    version int32)
}
    \end{lstlisting}
    Der FileService enthält Referenzen zu allen aktiven Listener.
    Nachdem ein FileHandle erstellt oder bearbeitet wurde, werden alle Listener durchlaufen.\\
    Im \acrshort{dmf}-\acrshort{lsp}-Server ist nur ein FileChangeListener implementiert:
    \subparagraph{Der DiagnosticsService}
    übermittelt die aktuellen Fehler in der Modelldatei an den Client.
    Dafür werden alle ErrorElemente in die Diagnostic Strukturen des \acrshort{lsp}s übersetzt.
    Schließlich werden die Daten mithilfe einer Request-Nachricht für die Methode ``textDocument/publishDiagnostics'' an den Client übermittelt.

    \subsubsection{SemanticTokensService}
    Um ein schnelles Verständnis einer Datei zu ermöglichen, ist die Einfärbung der Syntax und Semantik wichtig.
    Dafür stellt das \acrshort{lsp} die Möglichkeit bereit semantische Tokens zu übermitteln.\\
    Ein Token bezieht sich immer auf einen Bereich im Sourcecode und übermittelt einen Tokentyp und eine Auswahl der Tokenmodifikatoren.
    Die Tokentypen und Tokenmodifikatoren werden während der Initialisierung übermittelt.
    Der Client kann auch während der Initialisierung angeben, ob überlappende Bereich für Token unterstützt werden.
    Wenn die neuen semantischen Tokens übermittelt werden, werden die Tokens mithilfe einer Zahlenfolge codiert.
    \begin{center}
        \begin{tabular}{| c | c | m{21.5em} |}
            \hline
            \makecell{Index in der\\ Zahlenfolge\\für Token mit \\Index i} & Name & Erklärung \\
            \hline
            5*i & deltaLine & Die Zeilen zwischen dem letzten Token und diesem Token. \\
            \hline
            5*i+1 & deltaStart & Die Zeichen zwischen dem letzten Token und diesem Token.
            Relativ zu 0, falls der aktuelle Token in einer neuen Zeile ist. \\
            \hline
            5*i+2 & length & Die Länge des Tokens. \\
            \hline
            5*i+3 & tokenType & Index des Typs des Tokens in der Semantik Token Typ Legende. \\
            \hline
            5*i+4 & tokenModifiers & Zahl deren Bits als Wahrheitswerte für jeden Modifikator aus der Legende der Semantik Token Modifikatoren.
            Der erste Bit (0b00000001) steht dabei für den ersten Modifikator. \\
            \hline
        \end{tabular}
    \end{center}


    \paragraph{Der Legende für Semantik Token Typen}
    \begin{center}
        \begin{tabular}{|c|c|p{\mycolwidth}|}
            \hline
            Index & Token Typ & Verwendung im DMF \\
            \hline
            0 & namespace & Für den Namen eines Packages.\\
            \hline
            1 & type & \makecell[l]{Für das AST-Element ``reftype''.\\Für den Override Wert des Java Typens.}\\
            \hline
            2 & class & Für den Override Wert der Java Klasse, der Oberklasse und der implementierten Interfaces.\\
            \hline
            3 & enum & Für den Namen eines Enums.\\
            \hline
            4 & interface & Für den Namen eines Interfaces.\\
            \hline
            5 & struct & Für den Namen eines Structs oder einer Entity.\\
            \hline
            6 & parameter & Für die Parameter einer Funktion.\\
            \hline
            7 & variable & \makecell[l]{Für die Namen von Argumenten, Referenzen und \\MultiReferenzen.\\Für die Namen der Variablen im Entity Identifier.\\Für den Override Wert des Java Namens.}\\
            \hline
            8 & property &\\
            \hline
            9 & number & Für alle Zahlenwerte in einer Enum-Konstante.\\
            \hline
            10 & enumMember & Für den Namen einer Enum-Konstante.\\
            \hline
            11 & function & Für den Namen einer Funktion.\\
            \hline
            12 & comment & Für alle Kommentare.\\
            \hline
            13 & keyword & Für alle Keywords.\\
            \hline
            14 & string & Für alle Strings außer den Werten im Override.\\
            \hline
            15 & modifier & \\
            \hline
            16 & decorator & Für den Override Wert der Java Annotations.\\
            \hline
        \end{tabular}
    \end{center}
%TODO Modifier
    \paragraph{Generierung der Semantik Tokens}\mbox{}\\
    Die Semantik Tokens werden mithilfe zweier Algorithmen generiert.
    \subparagraph{Mithilfe des \acrshort{ast}s} werden alle \acrshort{ast}-Elemente durchlaufen.
    Werden Elemente erreicht, deren semantischer Token allein am \acrshort{ast} Element bestimmt werden kann, so werden die Token generiert.
    \subparagraph{Mithilfe des semantischen Modells} werden die restlichen Token bestimmt.
    Dies ist möglich da die Elemente im semantischen Modell Referenzen zum \acrshort{ast} beinhalten.
    Dazu gehören die Namen der verschiedenen Inhalte, wie z.B.\ von Argumenten.
    Bei diesen Namen wird das gleiche \acrshort{ast}-Element genutzt, wodurch sie nur das semantische Parsen unterscheidbar sind.

    \subparagraph{Die Codierung der semantischen Tokens} muss zunächst die generierten Tokens sortieren, da keine Garantie für die richtige Reihenfolge durch die beiden Algorithmen existiert.
    \begin{lstlisting}[language=Go,label={lst:sortSemanticTokens}]
slices.SortFunc(semanticElements, func(a, b *semanticElement) int {
    if a.line == b.line {
        return cmp.Compare(a.start, b.start)
    }
    return cmp.Compare(a.line, b.line)
})
    \end{lstlisting}
    Nach der Sortierung können die Tokens durchlaufen werden.
    \begin{lstlisting}[language=Go]
data := make([]uint32, len(semanticElements)*5)

lastLine := uint32(0)
lastStart := uint32(0)

for i, element := range semanticElements {
    line := element.line
    start := element.start
    length := element.length
    tokenType := element.tokenType
    tokenModifiers := element.tokenModifiers
    if line < lastLine {
        // Tokens must be sorted by line and character
        continue
    }

    if line == lastLine && start < lastStart {
        // Tokens must be sorted by line and character
        continue
    }

    // Calculate delta encoding
    deltaLine := line - lastLine
    deltaStart := uint32(0)
    if line == lastLine {
        deltaStart = start - lastStart
    } else {
        deltaStart = start
    }

    data[i*5] = deltaLine
    data[i*5+1] = deltaStart
    data[i*5+2] = length
    data[i*5+3] = tokenType
    data[i*5+4] = tokenModifiers

    lastLine = line
    lastStart = start
}
    \end{lstlisting}
    Beim Durchlaufen wird ein Integer-Slice erstellt. \\
    Für jeden Token werden die 5 Zahlen nach dem Protokoll hinzugefügt. \\
    Überschneiden sich Token oder sind nicht in der richtigen Reihenfolge werden die Token ignoriert. \\
    Die aktuelle Zeile und Spalte im Text wird nach jedem Token aktualisiert.

    \subsubsection{SelectionRangeService}
    Der SelectionRangeService implementiert die \acrshort{lsp}-Methode ``textDocument/selectionRange''.
    Diese wird dafür genutzt, dass bei der Auswahl immer der richtige Text ausgewählt wird. \\
    Die Bereiche werden im Service mithilfe des \acrshort{ast}s berechnet.
    \begin{lstlisting}[language=Go]
package selectionRangeService
func (s *SelectionRangeService) computeSelectionRangeForPosition(
        content fileService.FileContent,
        position protokoll.Position) *selectionRange.SelectionRange {
	finder := util.NewNodeFinder([]byte(content.Content))
	node := finder.FindSmallestNodeAroundPosition(content.Ast, position)
	var currentRange *selectionRange.SelectionRange
	for node != nil {
		newRange := &selectionRange.SelectionRange{
			Range: protokoll.ToRange(node.Range()),
		}
		newRange.Parent = currentRange
		currentRange = newRange

		node = node.Parent()
	}
	return currentRange
}
    \end{lstlisting}
    Es wird zunächst das kleinste Element(das ``Blatt'' des Baums) des \acrshort{ast}s für die Position Code bestimmt.\\
    Nun wird durch die Schichten des \acrshort{ast}s bis zur Wurzel iteriert.
    Bei jedem Element eine neue ``SelectionRange'' Instanz  angelegt.
    Dieses ``SelectionRange'' enthält immer den Bereich und eine Referenz zu der vorherigen ``SelectionRange''.
    Durch die Referenz wird die Baum-Struktur auch im Ergebnis übermittelt und es müssen weniger Anfragen an den Server gesendet werden.\\

    Ein Request kann mehrere Positionen enthalten.
    Deshalb wird der Algorithmus für jede Position ausgeführt.
\end{document}