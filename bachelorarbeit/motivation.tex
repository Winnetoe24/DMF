\documentclass{article}
\begin{document}
\section{Motivation}
In der Entwicklung von Enterprise Software werden sehr häufig Modelle zentral angelegt und aus dem Modell Code generiert. 
Es jedoch keine Frameworks welche eine Flexibilität bieten, die einen immer die beste Wahl einer Technologie zulässt.
\subsection{EMF}
EMF ist ein häufig Frameowkr zur Modellierung von Modellen in Java. EMF bietet dabei jedoch keine Wahl bei der IDE oder der Programmiersprache.
Dies führt dazu, dass Projekte und ganze Firmen bei ihren bisherigen Technologien stehen bleiben. Es wird bei Neuentwicklungen nicht mehr die gefragt, was wären die besten Technologien um das Problem zu lösen, sondern es wird gefragt, wie lösen wir das mit unserer bisherigen Architektur. 
\subsection{Effekte eines unflexiblen Frameworks}
Dieser fehlerhafte Ansatz schädigt das Projekt auf mehreren Ebenen:
1. Wissen und Erfahrung Konzentrierung
Da nur eine Archtitektur in Betracht gezogen wird, hat jedes Mitglied des Teams nur Erfahrung mit der aktuellen Architektur und jegliche Erfahrung mit anderen Technologien verfällt mit der Zeit. Dies schränkt die die Perspektiven auf Probleme sehr stark ein und macht einen Wechsel sehr aufwendig.
2. Schrinkender Bewerberanzahl
Da nur Bewerber für die gewählten Technologien in Betracht gezogen werden, verringert sich die Anzahl stark. Der Effekt wird verstärkt, wenn die Technologien als veraltet gelten. Eine kleinere Bewerberanzahl zwingt Unternehmen auch Bewerber, die anderfalls nicht beachtet worden wären, in Betracht zu ziehen. Dies führt zu weiteren Negativen Effekten, da einige schelchte Angestellte die Produktivität vieler guter Angestellter stark senken können.
3. Anfälligkeit gegenüber Sicherheitslücken
Eine starke Festlegung auf Technologien führt dazu, dass Sicherheitslücken gleich jedes Projekt betreffen. So könnten bei einer Zero-Day-Lücke direkt mehere Schicht im "Schweizer Käse Modell" (Source suchen)  wegfallen.
Diese Anffälligkeit wird stark erhöht sobald eine Technologie nicht mehr aktiv weiterentwickelt wird. Dies führt häufig dazu, dass andere Updates auch nicht genutzt werden können.
\end{document}